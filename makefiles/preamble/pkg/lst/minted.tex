%% Работа с листингами

%% Для листингов, сгенерённых программой pygment
\InputIfFileExists{lst.tex}{}{}

%% Need pygment <http://pygments.org/> <http://pypi.python.org/pypi/Pygments>
\usepackage[
newfloat,
%% Counts follows sections' numbering
% section,
%% Counts follows chapters' numbering
chapter,
]
{minted}

%% Поддержка русских букв {{{

\makeatletter
% Not needed for minted v2.0
\@ifpackagelater{minted}{2015/01/31}
{}
{
\ifdefined\minted@write@detok\@empty\else
\newcommand{\minted@write@detok}[1]{%
  \immediate\write\FV@OutFile{\detokenize{#1}}}%

\newcommand{\minted@FVB@VerbatimOut}[1]{%
  \@bsphack
  \begingroup
    \FV@UseKeyValues
    \FV@DefineWhiteSpace
    \def\FV@Space{\space}%
    \FV@DefineTabOut
    %\def\FV@ProcessLine{\immediate\write\FV@OutFile}% %Old, non-Unicode version
    \let\FV@ProcessLine\minted@write@detok %Patch for Unicode
    \immediate\openout\FV@OutFile #1\relax
    \let\FV@FontScanPrep\relax
%% DG/SR modification begin - May. 18, 1998 (to avoid problems with ligatures)
    \let\@noligs\relax
%% DG/SR modification end
    \FV@Scan}
    \let\FVB@VerbatimOut\minted@FVB@VerbatimOut

\renewcommand\minted@savecode[1]{
  \immediate\openout\minted@code\jobname.pyg
  \immediate\write\minted@code{\expandafter\detokenize\expandafter{#1}}%
  \immediate\closeout\minted@code}
\fi
}
\makeatother

%%}}}
%% Fix minted red box around some characters {{{

\usepackage{xpatch}
\makeatletter
\AtBeginEnvironment{minted}{\dontdofcolorbox}
\def\dontdofcolorbox{\renewcommand\fcolorbox[4][]{##4}}
\xpatchcmd{\inputminted}{\minted@fvset}{\minted@fvset\dontdofcolorbox}{}{}
\xpatchcmd{\mintinline}{\minted@fvset}{\minted@fvset\dontdofcolorbox}{}{}
\makeatother

%%}}}

%% Выравнивание по центру
%\RecustomVerbatimEnvironment{Verbatim}{BVerbatim}{}

%% Use color style
\ifdefined\printable
%% Версия для печати
\usemintedstyle{bw}
\else
\usemintedstyle{emacs}
\fi

%% Переносить строки
\setminted{breaklines=true}
%% Включает нумерацию строк
\setminted{linenos=true,
  numberblanklines=false,
}
%% Автоматически удаляет из кода все обычные пробельные символы
\setminted{autogobble=true}
%% Автоматически разрывать длинные строки
\setminted{breakanywhere=true}
%% Draws two lines, one on top and one at the bottom of the code to frame it
%% Other possible values are leftline, topline, bottomline and single.
% \setminted{frame=lines}
% \setminted{framesep=2mm}
%% Установить цвет фона листинга
% \usepackage[svgnames,dvipsnames]{xcolor}
% \setminted{bgcolor=LightGray}
%% Включить обычный математический режим внутри комментариев
\setminted{mathescape=true}
%% The line spacing of the code set to 1.2
\setminted{baselinestretch=1.2}

%% Change overall font size of minted
\setminted{fontsize=\footnotesize}

%% Set names for the English language
\addto\captionsenglish{%
  \providecommand\listingname{Listing}%
  \providecommand\listlistingname{List of listings}%
}
%% Set names for the Russian language
\addto\captionsrussian{%
  \providecommand\listingname{Листинг}%
  \providecommand\listlistingname{Список листингов}%
%  \renewcommand\listingscaption{Листинг}%
%  \renewcommand\listoflistingscaption{Список листингов}%
}

%%% Local Variables:
%%% mode: latex
%%% coding: utf-8-unix
%%% End:
